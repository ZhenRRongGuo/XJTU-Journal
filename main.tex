\documentclass{XJTU-Journal}
%%%%%%%%%%%%%%%%%%%%%%%%%% 基本信息 %%%%%%%%%%%%%%%%%%%%%%%%%%
\titlename{文章中文名称}%文章中文名称
\titlenameEng{The Title of the Article}%文章英文名称
\stuname{中文姓名}%姓名
\stunameEng{English name}
\stuIns{(1. 西安交通大学能源与动力工程学院, 710049, 西安)}%机构
\stuInsEng{(1. School of Energy and Power Engineering, Xi'an Jiaotong University, Xi'an 710049, China)}
\addbibresource{ref.bib} % 导入参考文献,使用biblatex生成,由biber编译
\raggedbottom
\begin{document}

\TheHead{%中文摘要
    中文摘要中文摘要中文摘要中文摘要中文摘要中文摘要中文摘要中文摘要中文摘要中文摘要中文摘要中文摘要中文摘要中文摘要中文摘要中文摘要中文摘要中文摘要中文摘要中文摘要中文摘要中文摘要中文摘要中文摘要中文摘要中文摘要中文摘要中文摘要中文摘要中文摘要中文摘要中文摘要中文摘要中文摘要中文摘要中文摘要中文摘要中文摘要中文摘要中文摘要中文摘要中文摘要中文摘要中文摘要中文摘要中文摘要中文摘要中文摘要中文摘要中文摘要中文摘要中文摘要中文摘要中文摘要中文摘要中文摘要中文摘要中文摘要中文摘要中文摘要中文摘要中文摘要中文摘要中文摘要中文摘要中文摘要中文摘要中文摘要中文摘要中文摘要中文摘要中文摘要中文摘要中文摘要中文摘要中文摘要中文摘要中文摘要中文摘要中文摘要中文摘要中文摘要中文摘要中文摘要中文摘要中文摘要中文摘要中文摘要中文摘要}%
    {中文关键词;中文关键词}
    {English Abstract English Abstract English Abstract English Abstract English Abstract English Abstract English Abstract English Abstract English Abstract English Abstract English Abstract English Abstract English Abstract English Abstract English Abstract English Abstract English Abstract English Abstract English Abstract English Abstract English Abstract English Abstract English Abstract English Abstract English Abstract English Abstract English Abstract English Abstract English Abstract English Abstract English Abstract English Abstract English Abstract English Abstract English Abstract English Abstract English Abstract English Abstract English Abstract English Abstract English Abstract English Abstract English Abstract English Abstract English Abstract English Abstract English Abstract English Abstract English Abstract English Abstract English Abstract English Abstract English Abstract English Abstract English Abstract English Abstract English Abstract English Abstract}% 英文摘要
    {English Keywords; English Keywords}

前言前言前言前言前言前言前言前言前言前言前言前言前言前言前言前言前言前言前言前言前言前言前言前言前言前言前言前言前言前言前言前言前言前言前言前言前言前言前言前言前言前言前言前言前言前言前言前言前言前言前言前言前言前言前言前言前言前言前言前言前言前言前言前言前言前言前言前言前言前言前言前言前言前言前言前言前言前言前言前言前言前言前言前言前言前言前言前言前言前言前言前言前言前言前言前言前言前言前言前言。

前言前言前言前言前言前言前言前言前言前言前言前言前言前言前言前言前言前言前言前言前言前言前言前言前言前言前言前言前言前言前言前言前言前言前言前言前言前言前言前言前言前言前言前言前言前言前言前言前言前言前言前言前言前言前言前言前言前言前言前言前言前言前言前言前言前言前言前言前言前言前言前言前言前言前言前言前言前言前言前言前言前言前言前言前言前言前言前言前言前言前言前言前言前言前言前言前言前言前言前言。

前言前言前言前言前言前言前言前言前言前言前言前言前言前言前言前言前言前言前言前言前言前言前言前言前言前言前言前言前言前言前言前言前言前言前言前言前言前言前言前言前言前言前言前言前言前言前言前言前言前言前言前言前言前言前言前言前言前言前言前言前言前言前言前言前言前言前言前言前言前言前言前言前言前言前言前言前言前言前言前言前言前言前言前言前言前言前言前言前言前言前言前言前言前言前言前言前言前言前言前言。

\pagestyle{headfootMain} %本命令用于设置非首页的页眉页脚,请手动设置这一行命令出现在第二页PDF中,否则会影响首页页眉页脚
\section{一级标题}
正文1正文1正文1正文1正文1正文1正文1正文1正文1正文1正文1正文1正文1正文1正文1正文1正文1正文1正文1正文1正文1正文1正文1正文1正文1正文1正文1正文1正文1正文1正文1正文1正文1正文1正文1正文1正文1正文1正文1正文1正文1正文1正文1正文1正文1正文1正文1正文1正文1正文1正文1正文1正文1正文1正文1正文1正文1正文1正文1正文1正文1正文1正文1正文1正文1正文1正文1\cite{Chugh2018}正文1正文1正文1正文1正文1


\subsection{二级标题}
正文2正文2正文2正文2正文2正文2正文2正文2正文2正文2正文2正文2正文2正文2正文2正文2正文2正文2正文2正文2正文2正文2正文2正文2正文2正文2正文2正文2正文2正文2正文2正文2正文2正文2正文2正文2正文2正文2正文2正文2正文2\cite{Coello2005,Zitzler2000,Zitzler2004}正文2正文2正文2正文2正文2正文2正文2正文2正文2正文2正文2正文2正文2正文2正文2正文2正文2正文2正文2正文2正文2正文2正文2正文2正文2正文2正文2正文2正文2正文2正文2

\subsubsection{三级标题}
正文3正文3正文3正文3正文3正文3正文3正文3正文3正文3正文3正文3正文3正文3正文3正文3正文3正文3正文3正文3正文3正文3正文3正文3正文3正文3正文3正文3正文3正文3正文3正文3正文3正文3正文3正文3正文3正文3正文3正文3正文3正文3正文3正文3正文3正文3正文3正文3正文3正文3正文3正文3正文3正文3正文3正文3正文3正文3正文3正文3正文3正文3正文3正文3正文3正文3正文3正文3正文3正文3正文3正文3

\section{模板使用}
\subsection{交叉引用}
模板使用 biblatex 编译参考文献,默认采用顺序编码制,由biber编译,同时注意需要在导言区导入参考.bib文件。引用参考文献请使用 \verb|\cite{}| 命令,如\cite{Cheng2016}。

本模板提供了table环境下的tabularx环境生成三线表,使用\verb|\bicaption{中}{En}|生成双语标题,请使用符号Y实现居中对齐,此外符号X实现居左对齐,符号Z实现居右对齐。
\begin{table}[!htb]
    \centering
    \bicaption{三线表}{English Table Name}
    \label{tab:三线表}
    \belowrulesep=0pt
    \aboverulesep=0pt
    \begin{tabularx}{\linewidth}{YYY}
    \toprule
    c1 &c2 &c3 \\
    \midrule
    内容1 &内容1 &内容1\\
    内容1 &内容1 &内容1\\
    \bottomrule
    \end{tabularx}
\end{table}

图片同样使用\verb|\bicaption{中}{En}|生成双语标题。
\begin{figure}[!htb]
    \centering
    \includegraphics[width=0.8\linewidth]{material/texlive吉祥物.png}
    \bicaption{图片}{Image}
    \label{fig:图片}
\end{figure}

\subsection{符号定义}
方便起见,本文定义了一些常用符号,可以直接调用,如表 \ref{tab:自定义符号} 所示。
\begin{table}[H]
    \centering
    \caption{自定义符号}
    \label{tab:自定义符号}
    \belowrulesep=0pt
    \aboverulesep=0pt
    \begin{tabularx}{\linewidth}{YYY}
    \toprule
    命令 & 符号 & 示例 \\
    \midrule
    \verb|\cel| &$\cel$ & $5\cel$ \\
    \verb|\diff| &$\diff$ & $\diff f = \diff x  + \diff y$ \\
    \verb|\Diff| &$\Diff$ & $\Diff f = \Diff x  + \Diff y$ \\
    \verb|\ve{}| &矢量符号 & $\ve{x}$\\
    \bottomrule
    \end{tabularx}
\end{table}
\subsection{流程图}
模板预设了流程图,可以直接使用tikzpicture绘制,如图 \ref{fig:NSGA3算法流程图} 所示。
\begin{figure}[!htp]
    \begin{tikzpicture}[node distance=0.5cm]
        %定义流程图具体形状
        \tikzstyle{every node}=[font=\small]
        \node[startstop](start){开始};
        \node[process, below of = start, yshift = -0.7cm](p1){生成初始种群};
        \node[process, below of = p1, yshift = -0.7cm](p2){选择、交叉、变异、生成子代};
        \node[process, below of = p2, yshift = -0.7cm](p3){合并父代和子代};
        \node[process, below of = p3, yshift = -0.7cm](p4){快速非支配排序};
        \node[decision, below of = p4, yshift = -1cm](d1){得到新父代};
        \node[process, right of = d1, xshift = 3.3cm](p5){基于参考点选择};
        \node[process, below of = d1, yshift = -1cm](p6){新父代};
        \node[decision, below of = p6, yshift = -0.7cm](d2){终止条件};
        % \node[coord,left of = d2, yshift = -0.3cm](c1){};
        \node[startstop, below of = d2, yshift = -0.7cm](stop){结束};
        \coordinate (point1) at (-3cm, -9cm);
        %连接具体形状
        \draw [arrow] (start) -- (p1);
        \draw [arrow] (p1) -- (p2);
        \draw [arrow] (p2) -- (p3);
        \draw [arrow] (p3) -- (p4);
        \draw [arrow] (p4) -- (d1);
        \draw (d1) -- node [above] {N} (p5);
        \draw [arrow] (p5) |- (p6);
        \draw [arrow] (d1) -- node [right] {Y} (p6);
        \draw [arrow] (p6) -- (d2);
        \draw (d2) -- node [above] {N} (point1);
        \draw (point1) |- (p2);
        \draw [arrow] (d2) -- (stop);
    \end{tikzpicture}
    \bicaption{NSGA-Ⅲ算法流程图}{Flowchart of the NSGA-Ⅲ algorithm}
    \label{fig:NSGA3算法流程图}
\end{figure}
流程图的具体绘制方法可参考 main.tex 文件此处的代码。

本文已导入tikz宏包,您可以使用tikz绘制其他图,具体绘制方法请参考tikz用法。
\section{结~论}
按照Word模板,结论二字中间有一个空格。

结论内容结论内容结论内容结论内容结论内容结论内容结论内容结论内容结论内容结论内容结论内容结论内容结论内容结论内容结论内容结论内容结论内容结论内容结论内容结论内容结论内容结论内容结论内容结论内容结论内容结论内容结论内容结论内容结论内容结论内容结论内容结论内容结论内容结论内容结论内容结论内容结论内容结论内容结论内容结论内容结论内容结论内容结论内容结论内容结论内容结论内容结论内容结论内容结论内容结论内容。
\balance %平衡双栏
\printbibliography % 参考文献列表
\end{document}