\documentclass{XJTU-Journal}
%%%%%%%%%%%%%%%%%%%%%%%%%% 基本信息 %%%%%%%%%%%%%%%%%%%%%%%%%%
\titlename{文章中文名称}%文章中文名称
\titlenameEng{The Title of the Article The Title of the Article The Title of the Article The Title of the Article The Title of the Article}%文章英文名称
\stuname{中文姓名}%姓名
\stunameEng{English name}
\stuIns{(1. 西安交通大学能源与动力工程学院, 710049, 西安)}%机构
\stuInsEng{(1. School of Energy and Power Engineering, Xi'an Jiaotong University, Xi'an 710049, China)}
%%%%%%%%%%%%%%%%%%%%%%%%%% 参考文献 %%%%%%%%%%%%%%%%%%%%%%%%%%
\addbibresource{ref.bib} % 导入参考文献,使用biblatex生成,由biber编译
%%%%%%%%%%%%%%%%%%%%%%%%%% 其他宏包 %%%%%%%%%%%%%%%%%%%%%%%%%%

\raggedbottom
\begin{document}

\TheHead{%中文摘要
    中文摘要中文摘要中文摘要中文摘要中文摘要中文摘要中文摘要中文摘要中文摘要中文摘要中文摘要中文摘要中文摘要中文摘要中文摘要中文摘要中文摘要中文摘要中文摘要中文摘要中文摘要中文摘要中文摘要中文摘要中文摘要中文摘要中文摘要中文摘要中文摘要中文摘要中文摘要中文摘要中文摘要中文摘要中文摘要中文摘要中文摘要中文摘要中文摘要中文摘要中文摘要中文摘要中文摘要中文摘要中文摘要中文摘要中文摘要中文摘要中文摘要中文摘要中文摘要中文摘要中文摘要中文摘要中文摘要中文摘要中文摘要中文摘要中文摘要中文摘要中文摘要中文摘要中文摘要中文摘要中文摘要中文摘要中文摘要中文摘要中文摘要中文摘要中文摘要中文摘要中文摘要中文摘要中文摘要中文摘要中文摘要中文摘要中文摘要中文摘要中文摘要中文摘要中文摘要中文摘要中文摘要中文摘要}%
    {中文关键词;中文关键词}
    {English Abstract English Abstract English Abstract English Abstract English Abstract English Abstract English Abstract English Abstract English Abstract English Abstract English Abstract English Abstract English Abstract English Abstract English Abstract English Abstract English Abstract English Abstract English Abstract English Abstract English Abstract English Abstract English Abstract English Abstract English Abstract English Abstract English Abstract English Abstract English Abstract English Abstract English Abstract English Abstract English Abstract English Abstract English Abstract English Abstract English Abstract English Abstract English Abstract English Abstract English Abstract English Abstract English Abstract English Abstract English Abstract English Abstract English Abstract English Abstract English Abstract English Abstract English Abstract English Abstract English Abstract English Abstract English Abstract English Abstract English Abstract English Abstract}% 英文摘要
    {English Keywords; English Keywords}

前言前言前言前言前言前言前言前言前言前言前言前言前言前言前言前言前言前言前言前言前言前言前言前言前言前言前言前言前言前言前言前言前言前言前言前言前言前言前言前言前言前言前言前言前言前言前言前言前言前言前言前言前言前言前言前言前言前言前言前言前言前言前言前言前言前言前言前言前言前言前言前言前言前言前言前言前言前言前言前言前言前言前言前言前言前言前言前言前言前言前言前言前言前言前言前言前言前言前言前言。

前言前言前言前言前言前言前言前言前言前言前言前言前言前言前言前言前言前言前言前言前言前言前言前言前言前言前言前言前言前言前言前言前言前言前言前言前言前言前言前言前言前言前言前言前言前言前言前言前言前言前言前言前言前言前言前言前言前言前言前言前言前言前言前言前言前言前言前言前言前言前言前言前言前言前言前言前言前言前言前言前言前言前言前言前言前言前言前言前言前言前言前言前言前言前言前言前言前言前言前言。

前言前言前言前言前言前言前言前言前言前言前言前言前言前言前言前言前言前言前言前言前言前言前言前言前言前言前言前言前言前言前言前言前言前言前言前言前言前言前言前言前言前言前言前言前言前言前言前言前言前言前言前言前言前言。

\pagestyle{headfootMain} %本命令用于设置非首页的页眉页脚,请手动设置这一行命令出现在第二页PDF中,否则会影响首页页眉页脚
\section{一级标题}
正文1正文1正文1正文1正文1正文1正文1正文1正文1正文1正文1正文1正文1正文1正文1正文1正文1正文1正文1正文1正文1正文1正文1正文1正文1正文1正文1正文1正文1正文1\cite{Chugh2018}正文1正文1正文1正文1正文1


\subsection{二级标题}
正文2正文2正文2正文2正文2正文2正文2正文2正文2正文2正文2正文2正文2正文2正文2正文2正文2正文2正文2正文2正文2正文2正文2正文2正文2正文2正文2正文2正文2正文2正文2正文2正文2正文2正文2正文2正文2正文2正文2正文2正文2\cite{Coello2005,Zitzler2000,Zitzler2004}正文2

\subsubsection{三级标题}
正文3正文3正文3正文3正文3正文3正文3正文3正文3正文3正文3正文3正文3正文3正文3正文3正文3正文3正文3正文3正文3正文3正文3正文3正文3正文3正文3正文3正文3正文3正文3正文3正文3正文3正文3正文3

\section{模板使用}
\subsection{交叉引用}
\subsubsection{参考文献}
模板使用 biblatex 编译参考文献,默认采用顺序编码制,由biber编译,同时注意需要在导言区导入参考文献.bib文件。引用参考文献请使用 \verb|\cite{}| 命令,如\cite{Cheng2016}。在.tex文件最后使用\verb|\printbibliography|命令生成参考文献列表。文档的编译顺序应当为 xelatex -> biber -> xelatex -> xelatex。

您也可以使用\verb|thebibliography|环境简单使用参考文献功能,此时只需要编译2次xelatex即可。但这种情况需要您手动设置参考文献引用格式,因此只建议您在文献数量很少的情况下使用。
\subsubsection{三线表}
本模板提供了table环境下的tabularx环境生成三线表,使用\verb|\bicaption{中}{En}|生成双语标题,请使用符号Y实现居中对齐,此外符号X实现居左对齐,符号Z实现居右对齐。符号XYZ会尽可能使表格的列均匀排布在一栏中,如果您的某一列内容较长,发生了自动换行,您也可以使用c、r、l符号来控制列的宽度,使某一列的列宽随内容长度变化。XYZ和crl符号可以混用。
\begin{table}[!htb]
    \centering
    \bicaption{三线表}{English Table Name}
    \label{tab:三线表}
    \belowrulesep=0pt
    \aboverulesep=0pt
    \begin{tabularx}{\linewidth}{YYY}
    \toprule
    c1 &c2 &c3 \\
    \midrule
    内容1 &内容1 &内容1\\
    内容1 &内容1 &内容1\\
    \bottomrule
    \end{tabularx}
\end{table}
\subsubsection{图片}
图片同样使用\verb|\bicaption{中}{En}|生成双语标题。图表的交叉引用请使用\verb|\ref{}|命令,如图 \ref{fig:图片},如表 \ref{tab:三线表},公式的交叉引用请使用\verb|\eqref{}|命令,如式 \eqref{eqn:示例} 所示。

一般情况下,当您中英文混排或中文数字混排时,比如这里的内容中文English中文English中文数字123中文数字123混排,\LaTeX 会自动在中英文和中文数字之间加入一个空格,使得排版更美观,这个空格也被称为“盘古之白”,它是一种排版规范。但当您使用 \verb|\ref{}| 和 \verb|\eqref{}| 命令时,\LaTeX 此时不会加入这些空格,比如这样如图\ref{eqn:示例}所示,这会显得内容非常拥挤,且不符合排版规范,因此建议您在使用 \verb|\ref{}| 和 \verb|\eqref{}| 命令时手动在命令前后添加一个空格,就像这样,如图 \ref{eqn:示例} 所示。
\begin{figure}[!htb]
    \centering
    \includegraphics[width=0.8\linewidth]{material/texlive吉祥物.png}
    \bicaption{图片}{Image}
    \label{fig:图片}
\end{figure}

\subsection{符号定义}
方便起见,本文定义了一些常用符号,可以直接调用,如表 \ref{tab:自定义符号} 所示。
\begin{table}[H]
    \centering
    \caption{自定义符号}
    \label{tab:自定义符号}
    \belowrulesep=0pt
    \aboverulesep=0pt
    \begin{tabularx}{\linewidth}{YYY}
    \toprule
    命令 & 符号 & 示例 \\
    \midrule
    \verb|\cel| &$\cel$ & $5\cel$ \\
    \verb|\diff| &$\diff$ & $\diff f = \diff x  + \diff y$ \\
    \verb|\Diff| &$\Diff$ & $\Diff f = \Diff x  + \Diff y$ \\
    \verb|\ii| &虚数符号$\ii$ & $\ii = \sqrt{-1}$ \\
    \verb|\e| &自然常数$\e$ & $\e^{\ii \pi} = -1$\\
    \verb|\ve{}| &矢量符号 & $\ve{x}$\\
    \bottomrule
    \end{tabularx}
\end{table}
\subsection{流程图}
模板预设了流程图,可以直接使用tikzpicture绘制,如图 \ref{fig:NSGA3算法流程图} 所示。流程图的具体绘制方法可参考 main.tex 文件此处的代码。
\begin{figure}[!htp]
    \begin{tikzpicture}[node distance=0.5cm]
        %定义流程图具体形状
        \tikzstyle{every node}=[font=\small]
        \node[startstop](start){开始};
        \node[process, below of = start, yshift = -0.7cm](p1){生成初始种群};
        \node[process, below of = p1, yshift = -0.7cm](p2){选择、交叉、变异、生成子代};
        \node[process, below of = p2, yshift = -0.7cm](p3){合并父代和子代};
        \node[process, below of = p3, yshift = -0.7cm](p4){快速非支配排序};
        \node[decision, below of = p4, yshift = -1cm](d1){得到新父代};
        \node[process, right of = d1, xshift = 3.3cm](p5){基于参考点选择};
        \node[process, below of = d1, yshift = -1cm](p6){新父代};
        \node[decision, below of = p6, yshift = -0.7cm](d2){终止条件};
        % \node[coord,left of = d2, yshift = -0.3cm](c1){};
        \node[startstop, below of = d2, yshift = -0.7cm](stop){结束};
        \coordinate (point1) at (-3cm, -9cm);
        %连接具体形状
        \draw [arrow] (start) -- (p1);
        \draw [arrow] (p1) -- (p2);
        \draw [arrow] (p2) -- (p3);
        \draw [arrow] (p3) -- (p4);
        \draw [arrow] (p4) -- (d1);
        \draw (d1) -- node [above] {N} (p5);
        \draw [arrow] (p5) |- (p6);
        \draw [arrow] (d1) -- node [right] {Y} (p6);
        \draw [arrow] (p6) -- (d2);
        \draw (d2) -- node [above] {N} (point1);
        \draw (point1) |- (p2);
        \draw [arrow] (d2) -- (stop);
    \end{tikzpicture}
    \bicaption{NSGA-Ⅲ算法流程图}{Flowchart of the NSGA-Ⅲ algorithm}
    \label{fig:NSGA3算法流程图}
\end{figure}


本文已导入tikz宏包,您可以使用tikz绘制其他图,具体绘制方法请参考tikz用法。

\subsection{列表环境}
如果您需要使用列表环境,本模板不建议使用enumerate环境,请您使用inparaenum环境,您可以参考main.tex此处的代码。

\begin{inparaenum}[(1)]%[(1)]用于给一级编号带上括号
\item 列表环境1。 % 请注意,每个item后需要加空行用来换行

\item 列表环境2。列表环境2。列表环境2。列表环境2。列表环境2。

\begin{inparaenum}[(a)]
    \item 您可以使用[(1)]用来给编号加上括号。
    
    \item 请注意,每个item后需要加空行用来换行。
    
    \item inparaenum环境可以嵌套使用。
    
\end{inparaenum}

\item 列表环境3。

\item 在inparaenum环境中,您可以嵌套使用浮动体或公式。如式 \eqref{eqn:示例} 所示。
\begin{equation}
    \sum_{i=1}^{n} f(n) = \frac{n(n+1)}{2}
    \label{eqn:示例}
\end{equation}
\end{inparaenum}

\subsection{通栏内容}
如果您有一个较宽的图片、表格或其他内容需要通栏排版(跨双栏),您可以使用以下方法。

\begin{inparaenum}%[]
    \item 对于图片,可以使用figure*环境,此时图片会优先排版在当前位置的下一页的顶端,如: \verb|\begin{figure*}[!htb]...\end{figure*}|,而不是紧跟内容排版,请参考第 \ref{节:通栏呈现结果} 节;
    
    \item 对于表格,可以使用table*环境,此时表格会优先排版在当前位置的下一页的顶端,如: \verb|\begin{table*}[!htb]...\end{table*}|,而不是紧跟内容排版,请参考第 \ref{节:通栏呈现结果} 节;
    
    \item 可以使用 strip 环境(模板已导入相关宏包,可以直接使用该环境)插入通栏文字,但请注意strip环境中无法放入浮动体。
    
\end{inparaenum}

双栏文章内使用通栏排版图表时无法做到让图表紧跟内容展示,这个到目前仍然没有一个比较好的解决方案。

对于图片而言,可以结合strip环境做到紧跟内容排版,但是无法支持双语标题,具体使用方法可以参考 \ref{节:通栏呈现结果} 节。

\section{其他注意事项}
本模板只能自用,比如您可以使用这个模板编译好PDF,从而提交课程作业。如果您需要向交大学报投稿,请您按照官方要求提交符合要求的文章。

由于本模板只能自用,因此没有提供更精细的功能,如只支持单作者,DOI、文章编号留空等。如果您有进一步需求,请自行修改模板。

\section{结~论}
按照Word模板,结论二字中间有一个空格。

结论内容结论内容结论内容结论内容结论内容结论内容结论内容结论内容结论内容结论内容结论内容结论内容结论内容结论内容结论内容结论内容结论内容结论内容结论内容结论内容结论内容结论内容结论内容结论内容结论内容结论内容结论内容结论内容结论内容结论内容结论内容结论内容结论内容内容结论内容结论内容结论内容结论内容结论内容结论内容结论内容结论内容结论内容结论内容结论内容结论内容结论内容结论内容。

\section{通栏呈现结果} \label{节:通栏呈现结果}
一大段文字一大段文字一大段文字一大段文字一大段文字一大段文字一大段文字一大段文字一大段文字一大段文字一大段文字一大段文字一大段文字一大段文字一大段文字一大段文字一大段文字一大段文字一大段文字一大段文字一大段文字一大段文字一大段文字一大段文字一大段文字一大段文字一大段文字一大段文字一大段文字一大段文字一大段文字一大段文字一大段文字一大段文字一大段文字一大段文字一大段文字一大段文字一大段文字一大段文字一大段文字一大段文字一大段文字一大段文字一大段文字一大段文字一大段文字一大段文字一大段文字一大段文字一大段文字一大段文字一大段文字一大段文字一大段文字一大段文字一大段文字一大段文字一大段文字一大段文字一大段文字一大段文字一大段文字一大段文字一大段文字一大段文字一大段文字一大段文字

这里分别使用table*和figure*环境插入通栏图表,如图 \ref{tab:使用table*环境排版通栏表}、表 \ref{fig:使用figure*环境排版通栏图片} 所示,可以看到,它们都出现在了下一页。

\begin{table*}[!htp]
    \centering
    \bicaption{使用table*环境排版通栏表}{English Table Caption}
    \label{tab:使用table*环境排版通栏表}
    \belowrulesep=0pt
    \aboverulesep=0pt
    \begin{tabularx}{\linewidth}{YYYY}
    \toprule
    1 & 2 &3 &4\\
    \midrule
    1 & 2 &3 &4\\
    1 & 2 &3 &4\\
    \bottomrule
    \end{tabularx}
\end{table*}

\begin{figure*}[!htb]
    \centering
    \includegraphics[width=0.6\linewidth]{material/texlive吉祥物.png}
    \bicaption{使用figure*环境排版通栏图片}{English Figure Caption}
    \label{fig:使用figure*环境排版通栏图片}
\end{figure*}

一大段文字一大段文字一大段文字一大段文字一大段文字一大段文字一大段文字一大段文字一大段文字一大段文字一大段文字一大段文字一大段文字一大段文字一大段文字一大段文字一大段文字

这里插入使用strip环境插入一个通栏图片,如图 \ref{fig:strip通栏图片} 所示,这里不支持双语标题,而且似乎会引起附近段落的弹性拉伸。

\begin{strip}
    \centering\includegraphics[width=0.6\linewidth]{material/texlive吉祥物.png}
    \captionof{figure}{使用strip环境排版通栏图片,但不支持双语标题}
    \label{fig:strip通栏图片}
\end{strip}


一大段文字一大段文字一大段文字一大段文字一大段文字一大段文字一大段文字一大段文字一大段文字一大段文字一大段文字一大段文字一大段文字一大段文字一大段文字一大段文字一大段文字一大段文字一大段文字一大段文字一大段文字一大段文字一大段文字一大段文字一大段文字一大段文字一大段文字一大段文字一大段文字一大段文字一大段文字一大段文字一大段文字一大段文字一大段文字一大段文字一大段文字一大段文字一大段文字一大段文字一大段文字一大段文字一大段文字一大段文字一大段文字一大段文字一大段文字一大段文字一大段文字一大段文字一大段文字一大段文字一大段文字一大段文字一大段文字一大段文字一大段文字一大段文字一大段文字一大段文字一大段文字一大段文字一大段文字一大段文字一大段文字

\balance %平衡双栏
\printbibliography % 参考文献列表
\end{document}